% This template has been downloaded from:
% http://www.latextemplates.com
%
% Original author:
% Ted Pavlic (http://www.tedpavlic.com)
%
% Modified by:
% Charles Newey (http://assemblyco.de)
%----------------------------------------

% Declare document
\documentclass{article}

% Packages
\usepackage{fancyhdr} % Required for custom headers
\usepackage{lastpage} % Required to determine the last page for the footer
\usepackage{extramarks} % Required for headers and footers
\usepackage{graphicx} % Images
\usepackage{tabularx} % Tables
\usepackage[table]{xcolor} % Table colours
\usepackage[colorlinks]{hyperref} % For URLs
\usepackage[T1]{fontenc} % Support symbols like < and >
\usepackage{lmodern} % Format symbols properly
\usepackage{listings} % Code highlighting

% Colours
\definecolor{negligible}{rgb}{0.55, 0.71, 0.0}
\definecolor{acute}{rgb}{1.0, 0.75, 0.0}
\definecolor{severe}{rgb}{1.0, 0.49, 0.0}
\definecolor{critical}{rgb}{0.9, 0.17, 0.31}

% Listings
\definecolor{javared}{rgb}{0.6,0,0} % for strings
\definecolor{javagreen}{rgb}{0.25,0.5,0.35} % comments
\definecolor{javapurple}{rgb}{0.5,0,0.35} % keywords
\definecolor{javadocblue}{rgb}{0.25,0.35,0.75} % javadoc

\lstset{language=Java,
	basicstyle=\ttfamily,
	keywordstyle=\color{javapurple}\bfseries,
	stringstyle=\color{javared},
	commentstyle=\color{javagreen},
	morecomment=[s][\color{javadocblue}]{/**}{*/},
	numbers=left,
	numberstyle=\tiny\color{black},
	stepnumber=2,
	numbersep=10pt,
	tabsize=4,
	showspaces=false,
	showstringspaces=false}

% Margins
\topmargin=-0.45in
\evensidemargin=0in
\oddsidemargin=0in
\textwidth=6.5in
\textheight=9.0in
\headsep=0.25in
\linespread{1} % Line spacing

% Other setup
\pagestyle{fancy}
\renewcommand\headrulewidth{0.4pt} % Size of the header rule
\renewcommand\footrulewidth{0.4pt} % Size of the footer rule
\setlength\parindent{0pt} % Removes all indentation from paragraphs
\renewcommand{\refname}{} % Removes bibliography title

% Set up constants
\newcommand{\address}{
\small{
	\begin{tabular}{ l}
		Department of Computer Science, \\
		Llandinam Building, \\
		Aberystwyth University, \\
		Aberystwyth, \\
		Ceredigion, \\
		SY23 3DB \\
	\end{tabular}
	}
}

% Set up the header and footer
\lhead{\doctitle}										% Top left header
\chead{\version}											% Top center head
\rhead{\firstxmark \status}								% Top right header
\lfoot{\lastxmark \qanumber}								% Bottom left footer
\cfoot{Aberystwyth University/Computer Science}			% Bottom center footer
\rfoot{Page\ \thepage\ of\ \protect\pageref{LastPage}}	% Bottom right footer

% Set up title page
\title{
	\vspace{1.2in}
	\textmd{\textbf{\doctitle}} \\
	\vspace{0.1in}\large{\textit{\today}} \\
	\vspace{0.4in}
	{\bf{\qanumber}} \\ \vspace{0.4in}
	\version \\
	\status \\
	\vspace{0.4in}
}

\author{\authors}
\date{}


%----------------------------- UPDATE THESE FOR EACH DOCUMENT ------------------------------
\newcommand{\version}{Version: 1.0} %======================================================= DOC VERSION
\newcommand{\status}{Status: Release} %===================================================== DOC STATUS
\newcommand{\qanumber}{SE.10.TEMPLATE.1} %================================================== QA NUMBER
\newcommand{\doctitle}{Group 10 Design Specification} %======================================== DOC TITLE

%----------------------------- UPDATE THESE FOR EACH DOCUMENT ------------------------------
%=========================================================================================== VERSION HISTORY
\newcommand{\versionhistory}{
		\begin{tabularx}{\linewidth}{| p{2cm} | p{2cm} | p{2cm} | X | }
			\hline
			\bf{Author} & \bf{Date} & \bf{Version} & \bf{Change made} \\
			\hline
			CCN & 03/12/2013 & 1.0 & Initial build of document \\
			\hline
		\end{tabularx}
}

%---------------------------- UPDATE THESE FOR EACH DOCUMENT ------------------------------
%=========================================================================================== AUTHOR LIST
\newcommand{\authors}{
	\begin{tabular}{| l | l |}
		\hline
		\bf{Contributor Name} & \bf{Role} \\
		\hline
		Daniel Clark & Project Lead \\
		\hline
		Mark Lewis & QA Manager \\
		\hline
		Charles Newey & Deputy Project Lead \& Android Developer \\
		\hline
		Martin Ferris & Android Developer \\
		\hline
		Ashley Iles & Android Developer \\
		\hline
		Kenny Packer & Android Developer \\
		\hline
		Stephen McFarlane & Deputy QA \& Web Developer \\
		\hline
		Kieran Palmer & Web Developer \\
		\hline
	\end{tabular}
	% Don't edit this
	\\ \\ \\ \\ \\ \\
	\address \vline
	\hspace{0.15in} \copyright Copyright Group 10, 2013
	% Don't edit this
}

% Make title page, ToC and other introductory elements
\begin{document}
	\maketitle
	\newpage
	\tableofcontents
	\newpage

	% Begin the actual document
	%======================================================================================= DOCUMENT STARTS HERE
	\begin{section}{INTRODUCTION}
		\begin{subsection}{Purpose of This Document}
			The purpose of this document is to show that we have met the outlined objectives specified by the client.
		\end{subsection}
	
		\begin{subsection}{Scope}
			Scope
		\end{subsection}
		
		\begin{subsection}{Objectives}
			\begin{itemize}
				\item{Objective 1}
				\item{Objective 2}
			\end{itemize}
		\end{subsection}
	\end{section}
	
	\begin{section}{DECOMPOSITION DESCRIPTION}
		\begin{subsection}{Programs in System}
		The System that will be created will contain three systems; the Android app, a server and a web page. \\

		The Android side of the system will be used as the platform for the walking tour creator app. Within this app users can create a new tour, they will then be able to add locations to the tour. These locations will contain information such as the tours title, description, the way points of the location and possibly an image. Once the tour is saved and finished it can then be sent via Wi-Fi or internet connection on the device to the server. The data will be sent to the server as a Multimedia Internet Message Extension in a http POST.\\

		The server will be used to store all data about tours created on the Android app. On the server a database will be created and within the database a table for storing the tour's details. Once data is received a new record will be created and the data stored for easy re-use on the web page. To access this data the web page will use a php script with mySQL commands.\\

		The final program in the system is the web page, which will be used for displaying tours. On the website there will be a list of saved tours from which the user can click to view them. The tour will then be displayed on a map which will incorporate the use of the Google Maps API. By clicking on a way point the user will be able get more information depending on what was saved by the creator using the Android app.
		\end{subsection}
	\end{section}
	
	\begin{section}{SIGNIFICANT CLASSES}
		\begin{subsection}{StumblrData}
		This will be an abstract class which will contain variables and validation methods that other classes will use, such as short description and title.
		\end{subsection}

		\begin{subsection}{Waypoint}
		This class will handle all information and methods regarding locations on the route. A method will be used to create a waypoint containing a title, short description and an image. Coordinates will be stored in a linked list of type coordinate.
		\end{subsection}

		\begin{subsection}{Route}
		This class will contain all the methods to handle all parts of creating a route, such as adding waypoints to a linked list of type Waypoint. 
		\end{subsection}
	\end{section}
	
	\clearpage
	\begin{section}{INTERFACE DESCRIPTIONS}
		\begin{subsection}{Android Activities}
			\lstinputlisting[label=AbstractActivity,
			caption=AbstractActivity \{Abstract\}]{interfaces/AbstractActivity.java}
		\clearpage
			\lstinputlisting[label=DataEntryActivity,
			caption=DataEntryActivity \{Abstract\}]{interfaces/DataEntryActivity.java}
			\lstinputlisting[label=Home,
			caption=Home]{interfaces/Home.java}
		\clearpage
			\lstinputlisting[label=CreateRoute,
			caption=CreateRoute]{interfaces/CreateRoute.java}
		\clearpage
			\lstinputlisting[label=CreateWaypoint,
			caption=CreateWaypoint]{interfaces/CreateWaypoint.java}
		\clearpage
			\lstinputlisting[label=WaypointList,
			caption=WaypointList]{interfaces/WaypointList.java}
		\clearpage
			\lstinputlisting[label=FinishRoute,
			caption=FinishRoute]{interfaces/FinishRoute.java}
		\end{subsection}
			\clearpage
		\begin{subsection}{Data Classes}
			\lstinputlisting[label=StumblrData,
			caption=StumblrData]{interfaces/data/StumblrData.java}
		\clearpage
			\lstinputlisting[label=Route,
			caption=Route]{interfaces/data/Route.java}
		\clearpage
			\lstinputlisting[label=Waypoint,
			caption=Waypoint]{interfaces/data/Waypoint.java}
		\end{subsection}
	\end{section}

	% Include references here (edit the References.bib file)
	\nocite{LaTeXTemplate} % DO NOT EDIT
	\nocite{LaTeXListings}
	\nocite{LaTeXListings2}

	% Format bibliography/refs
	\newpage
	\begin{section}{REFERENCES}
		\bibliographystyle{acm}
		\bibliography{References}
	\end{section}
	
	\vspace{1cm}
	\begin{section}{VERSION HISTORY}
		\versionhistory
	\end{section}
\end{document}

%									Useful bits and pieces
%\begin{section}{section_name}								% Start section
%\end{section}												% End section
%\begin{center} \end{center}									% Center stuff
%\includegraphics[width=0.75\columnwidth]{example_figure}		% Insert image
%\pseudocode{filename}{caption}								% Insert highlighted code snippet
%\clearpage													% Clear page after section
%\url{http://www.google.com/} 								% Include URL
%\nocite{citationName}										% Cite to bibliography (but not to text)
%\cite{citationName}											% Include reference