% This template has been downloaded from:
% http://www.latextemplates.com
%
% Original author:
% Ted Pavlic (http://www.tedpavlic.com)
%
% Modified by:
% Charles Newey (http://assemblyco.de)
%----------------------------------------

% Declare document
\documentclass{article}

% Packages
\usepackage{fancyhdr} % Required for custom headers
\usepackage{lastpage} % Required to determine the last page for the footer
\usepackage{extramarks} % Required for headers and footers
\usepackage{graphicx} % Images
\usepackage{tabularx} % Tables
\usepackage[colorlinks]{hyperref} % For URLs
\usepackage[T1]{fontenc} % Support symbols like < and >
\usepackage{lmodern} % Format symbols properly

% Margins
\topmargin=-0.45in
\evensidemargin=0in
\oddsidemargin=0in
\textwidth=6.5in
\textheight=9.0in
\headsep=0.25in
\linespread{1} % Line spacing

% Other setup
\pagestyle{fancy}
\renewcommand\headrulewidth{0.4pt} % Size of the header rule
\renewcommand\footrulewidth{0.4pt} % Size of the footer rule
\setlength\parindent{0pt} % Removes all indentation from paragraphs
\renewcommand{\refname}{} % Removes bibliography title

% Set up constants
\newcommand{\address}{
\small{
	\begin{tabular}{ l}
		Department of Computer Science, \\
		Llandinam Building, \\
		Aberystwyth University, \\
		Aberystwyth, \\
		Ceredigion, \\
		SY23 3DB \\
	\end{tabular}
	}
}

% Set up the header and footer
\lhead{\doctitle}										% Top left header
\chead{\version}											% Top center head
\rhead{\firstxmark \status}								% Top right header
\lfoot{\lastxmark \qanumber}								% Bottom left footer
\cfoot{Aberystwyth University/Computer Science}			% Bottom center footer
\rfoot{Page\ \thepage\ of\ \protect\pageref{LastPage}}	% Bottom right footer

% Set up title page
\title{
	\vspace{1.2in}
	\textmd{\textbf{\doctitle}} \\
	\vspace{0.1in}\large{\textit{\today}} \\
	\vspace{0.4in}
	{\bf{\qanumber}} \\ \vspace{0.4in} % QA document number
	\version \\
	\status \\
	\vspace{0.4in}
}

\author{\authors}
\date{}


%----------------------------- UPDATE THESE FOR EACH DOCUMENT ------------------------------
\newcommand{\version}{Version: 1.0}
\newcommand{\status}{Status: Draft}
\newcommand{\qanumber}{SE.10.D1}
\newcommand{\doctitle}{Group 10 Project Plan}

\newcommand{\authors}{ % Include a table for authors
	\begin{tabular}{| l | l |}
		\hline
		\bf{Contributor Name} & \bf{Role} \\
		\hline
		Daniel Clark & Project Lead \\
		\hline
		Charles Newey & Deputy Project Lead \\
		\hline
		Mark Lewis & QA Manager \\
		\hline
		Ashley Iles & Android Developer \\
		\hline
		Kenny Packer & Android Developer \\
		\hline
		Stephen McFarlane & Web Developer \\
		\hline
	\end{tabular}
	% Don't edit this
	\\ \\ \\ \\ \\ \\
	\address \vline
	\hspace{0.15in} \copyright Copyright Group 10, 2013
	% Don't edit this
}

% Make title page, ToC and other introductory elements
\begin{document}
	\maketitle
	\newpage
	\tableofcontents
	\newpage

	% Begin the actual document
	%-------------------------------------- DOCUMENT STARTS HERE ------------------------------------
	\begin{section}{INTRODUCTION (\qanumber.01)}
		\begin{subsection}{Purpose of This Document}
			The purpose of this document is to show that we have met the outlined objectives specified by the client. 

The main objective is as follows: "Walking Tour Creator (WTC) is a computer-based system to compile data about a walking tour, and structure it in a database that can be used by a mobile application which guides people around the walk, showing them places of interest along the way. WTC will be able to build a number of related walks, showing the location and details of places of interest along with photos, audio, video, about each place. All textual data will be stored in English." \\

This document will show we have managed to simplify these into a set of manageable goals. Gantt charts and GitHub's contribution analysis functionality will be used in conjunction to show and monitor the progress of the project's major tasks and other milestones. We are also implementing a risk analysis system that should alert us to the various problems that we may face whilst completing our objectives, and also allow us to mitigate the risks involved.
		\end{subsection}
	
		\begin{subsection}{Scope}
			This document should take into account the specifications of the project. This document includes an overview of our proposed systems - which will encompass our choice of platforms, some high-level architectures and a description of prospective users. The document also contains; a use case diagram giving an overview of how the app and the web systems will interact, mock-ups and a description of the UI design and how both applications will interact with the user, a Gantt chart which displays the start and end dates for the main milestones of the project, and a risk analysis for the project.
		\end{subsection}
		
		\begin{subsection}{Objectives}
			These main objectives are to show our initial plans for the project. These goals include an overview of our proposed system, a set of use case diagrams, user interface designs, a Gantt chart, and a risk analysis.
			
			\begin{itemize}
				\item{Produce an overview of our proposed system, which matches our client's specifications.}
				\item{Produce a set of detailed use case diagrams to define the interactions with the system components and users and provide them with a concise explanation, along with example usage scenarios.}
				\item{Create a UI for the Android and web systems we will employ, with a succinct explanation for our chosen design.}
				\item{Employ the usage of a Gantt chart, to document our expected progress as a team.}
				\item{Highlight aspects of our project plan which may cause us difficulties in completing our assigned work in a timely manner.}
			\end{itemize}
		\end{subsection}
	\end{section}

	\newpage
	\begin{section}{SYSTEM OVERVIEW}
		\begin{subsection}{Platforms and High Level Architecture}
			\bf{Sample text}
		\end{subsection}
		
		\begin{subsection}{Target Users}
			\bf{Sample text}
		\end{subsection}
		
		\begin{subsection}{Client and Server Structure}
			\bf{Sample text}
			Include stuff about protocol, yada yada
		\end{subsection}
	\end{section}

	\begin{section}{USE CASES AND SCENARIOS}
		\begin{subsection}{Core Android Functionality}
			\begin{subsubsection}{Whole System}
				\bf{Sample text}
			\end{subsubsection}
			
			\begin{subsubsection}{Create Route}
				\bf{More text}
			\end{subsubsection}
		\end{subsection}
		
		\begin{subsection}{Core Website Functionality}
			\bf{Sample text}
		\end{subsection}
	\end{section}
	
	\newpage
	\begin{section}{RISK ANALYSIS}
		\begin{subsection}{Ongoing Tasks}
			\begin{tabularx}{\linewidth}{| p{3.5cm} | p{0.5cm} | p{0.5cm} | p{0.8cm} | X |}
				\hline
				\bf{Risk Event} & \bf{L} & \bf{M} & \bf{Risk} & \bf{Mitigation} \\
				\hline
				Team member absence & 0.6 & 0.3 & 0.18 & All team members to regularly check emails and the agreed online resources for meeting times. Being unaware of meetings is not a valid excuse. If a team member is unable to attend a meeting, the project lead (Daniel) must be notified as soon as they know they can't attend. \\
				\hline
				Project Lead absence & 0.6 & 0.5 & 0.30 & Meeting to go ahead as planned with Charlie (Deputy Project Lead) taking the meeting. \\
				\hline
				QA Manager absence & 0.6 & 0.3 & 0.18 & Meeting to go ahead as planned with Steve (Deputy QA Manager). \\
				\hline
				Unable to contact team member & 0.3 & 0.8 & 0.24 & Ensure that all team members regularly check emails and other agreed online resources, as well as checking meeting minutes so they are aware of any outstanding tasks/actions. Persistently being unreliable with result in a warning, and further action if necessary; e.g. carding or role reallocation. \\
				\hline
				Git failure & 0.3 & 1.0 & 0.3 & All work to be backed up regularly in several places in case of human error or Git failure. \\
				\hline
				Major illness or unexpected circumstances & 0.5 & 0.9 & 0.45 & Team members to be notified as soon as possible, in case any urgent tasks need to be re-assigned or completed by another team member. \\
				\hline
				Git conflict & 0.3 & 0.9 & 0.27 & All team members are to have read the information on the project wiki on Git conflicts. If a conflict is encountered, then it must be resolved immediately. If a conflict cannot be resolved easily, then an appropriate team member (Git expert/project lead) must be notified and the conflict must be resolved. Try to ensure an even task allocation, to avoid multiple team members working on the same code simultaneously. \\
				\hline
			\end{tabularx}
		\end{subsection}
		
		\begin{subsection}{Documentation}
			\begin{tabularx}{\linewidth}{| p{3.5cm} | p{0.5cm} | p{0.5cm} | p{0.8cm} | X |}
				\hline
				\bf{Risk Event} & \bf{L} & \bf{M} & \bf{Risk} & \bf{Mitigation} \\
				\hline
				Late submission & 0.4 & 0.8 & 0.32 & Deadlines for documentation to be brought forward to ensure that any future problems encountered will come to light within a reasonable time frame. If team members run into any problems, they are to alert the group so that a solution can be issued. \\
				\hline
				Inadequate quality submission & 0.4 & 1.0 & 0.4 & Documents to be checked by either of the QA managers or project leaders before submitting. If team members need help then they should ask the rest of the team for help. \\
				\hline
				Human error & 0.5 & 0.2 & 0.1 & All documents to be checked for spelling, grammar, logic, and clerical errors by the creator of each document and at least one other team member. \\
				\hline
				Loss of documentation & 0.4 & 1.0 & 0.4 & Documentation to be stored and versioned on Git and backed up individually by the document's creator. Each individual is responsible for their own documentation. \\
				\hline
			\end{tabularx}
		\end{subsection}
		
		\newpage
		\begin{subsection}{Software Development}
			\begin{tabularx}{\linewidth}{| p{3.5cm} | p{0.5cm} | p{0.5cm} | p{0.8cm} | X |}
				\hline
				\bf{Risk Event} & \bf{L} & \bf{M} & \bf{Risk} & \bf{Mitigation} \\
				\hline
				Project behind schedule & 0.5 & 1.0 & 0.5 & Development to begin as soon as possible. Charlie (lead developer) to delegate any appropriate outstanding tasks. Regular feedback to be given by Daniel (project lead) to ensure schedule is adhered to. \\
				\hline
				Parts of the project missing/incomplete & 0.4 & 1.0 & 0.4 & Daniel (project lead) and Charlie (lead developer) to delegate programming tasks appropriately. Constant testing by QA managers and project leaders to check if code is of a sufficient quality. \\
				\hline
			\end{tabularx}
		\end{subsection}
		
		\begin{subsection}{Risk Grade and Recommended Action Key}
			\begin{tabularx}{\linewidth}{| X | X | X | X | X | X | X | X |}
				\hline
				\bf{Risk Grade} & Less than negligible & Negligible & Acute & Severe & Critical & Catastrophic \\
				\hline
				\bf{Risk score} & < 0.2 & 0.2 - 0.39 & 0.4 - 0.59 & 0.6 - 0.79 & 0.8 - 0.99 & 1.0 \\
				\hline
				\bf{Action} & Tolerate & Tolerate & Tolerate or treat & Treat & Transfer & Terminate \\
				\hline
			\end{tabularx}
		\end{subsection}
	
	\end{section}

	% Include references here (edit the References.bib file)
	\nocite{LaTeXTemplate}

	% Format bibliography/refs
	\newpage
	\begin{section}{REFERENCES}
		\bibliographystyle{acm}
		\bibliography{References}
	\end{section}
\end{document}

%									Useful bits and pieces
%\begin{section}{section_name}								% Start section
%\end{section}												% End section
%\begin{center} \end{center}								% Center stuff
%\includegraphics[width=0.75\columnwidth]{example_figure}	% Insert image
%\pseudocode{filename}{caption}								% Insert highlighted code snippet
%\clearpage													% Clear page after section
%\url{http://www.google.com/} 								% Include URL
%\nocite{citationName}										% Cite to bibliography (but not to text)
%\cite{citationName}										% Include reference